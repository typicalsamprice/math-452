% -*- compile-command: "latexmk -pdf document.tex" -*-
\documentclass{article}

\input{template}
\togglefalse{paper}

\input{theoremboxes}

\author{Sam Price}
\date{}
\title{Math 452: Assignment 2}

\usepackage{dcolumn}
\newcolumntype{2}{D{.}{}{2.0}}

\begin{document}

\maketitle

\begin{enumerate}
  \item[6.3\rparen]
        \begin{itemize}
                % s(g) = e, g(s) = e
          \item Simply, $C(\gamma) = \set*{\eps, \sig, \gamma}$.
                The $\eps, \gamma$ by triviality, and $\sig$ by inspection.
          \item $C(\tau) = \set*{\eps, \tau}$.
        \end{itemize}
  \item[6.5\rparen] For $n > 2$, $Z(S_{n})$ is trivial and so $Z(S_{3}) = \set{\eps}$.
  \item[6.8\rparen] Show that for $n > 2$, $S_{n}$ is noncommutative.
        \begin{proof}
          Let $\sig, \tau$ be two involutions in $S_{n}$ for $n > 2$\, such that 1 is not a fixed
          point and $\sig(1) \ne \tau(1)$.
          Without loss of generality, we may declare $\sig(1) = 2$ and $\tau(1) = 3$.
          Then, $\sig(\tau(1)) \ne \tau(\sig(1))$ and as such $S_{n}$ is noncommutative.
        \end{proof}
  \item[6.19\rparen] Let $H = \set*{ \sig \in S_{5} \mid \sig(3) = 3 }$.
        Show that $H$ is a subgroup of $S_{5}$.
        \begin{proof}
          Suppose that $h, g \in H$ such that $h \circ g \notin H$. Then this implies $h(g(3)) \ne 3$.
          However, this means $h(3) \ne 3$ which is impossible.

          Not sure if this proof is accepted, but creating the isomorphism $H \isomorphism S_{4}$
          would be just as valid in my eyes; I am not sure the best way to go about
          making that stronger than ``just look at it, duh'' justification.
        \end{proof}
  \item[8.4\rparen]
        \begin{itemize}
          \item $U_{10} = \set*{a \in \bbZ_{10} \mid \gcd(a, 10) = 1}$.
          \item
                  \renewcommand{\arraystretch}{1.3}
                  \setlength\doublerulesep{0pt}
                  \begin{tabular}{|r||*{5}{2|}}
                    \hline
                    $\times$ & 1 & 3 & 7 & 9 \\ \hline\hline
                    1 & 1 & 3 & 7 & 9 \\ \hline
                    3 & 3 & 9 & 1 & 7 \\ \hline
                    7 & 7 & 1 & 9 & 3 \\ \hline
                    9 & 9 & 7 & 3 & 1 \\ \hline
                  \end{tabular}
          \item Clearly 1 is the identity, and every element has something whose product is 1.
                Also, we only see elements of $U_{10}$. Thus, we have identity, inverses and closure and so a group. We take associativity on faith so I do not mention it.
          \item The set of units is commutative as the underlying operation on the
                ring is commutative. Otherwise, one might see that every row is
                identical to its corresponding column in the table above.
        \end{itemize}
  \item[8.18\rparen] Show that if $g = g\inv$ for all $g \in G$, then $G$ is abelian.
        \begin{proof}
          Let $G$ be a group such that every element is its own inverse.
          Then take $h, g \in G$, and we see that $(hg)(gh) = 1$. Thus, $gh = (hg)\inv = hg$.
        \end{proof}
  \item[9.11\rparen]
        \begin{itemize}
          \item $\bbZ_{3}$ is generated by 1, and the canonical cyclic group of order 3.
          \item $\set{\eps, r_{120}, r_{240}}$ is isomorphic to $\bbZ_{3}$ and is
                so generated by $r_{120}$.
          \item This is also generated by $\sig$, as $\sig^{2} = \tau$ and $\sig^{3} = \eps$.
        \end{itemize}
  \item[9.13\rparen] As $\gcd(3, 4) = 1$ then it is a valid generator.
        Similarly by inspection $i$ generates $C$ in this case.
  \item[9.15\rparen] Calculate $\bbZ_{2} \times \bbZ_{2}$. Is it cyclic?
        No, as either $(1, 1)$ or $(1, 0)$ will be missed by any element in $\bbZ_{2} \times \bbZ_{2}$.
\end{enumerate}


\end{document}
