% -*- compile-command: "latexmk -pdf document.tex" -*-
\documentclass{article}

\author{Sam Price}
\title{MATH 452: Test 2}

\newif\ifprinted%
\printedtrue%
\input{preamble}

\begin{document}

\maketitle

\begin{enumerate}[start=1,label={(\arabic*)}]
  \item Choose \underline{two} to solve.

        \begin{enumerate}[start=1,label={(\alph*)}]
          \item Suppose that $\ord(g) = n$.
                Find the necessary and sufficient condition on $r, s \in \ZZ$ so that $\angles{g^{r}} \subset \angles{g^{s}}$.

                \begin{proposition}{}{}
                  If $g \in G$ has order $n$, then $\angles{g^{r}} \subset \angles{g^{s}}$ for integers $r, s$ if and only if $r \equiv sp \pmod{n}$ for some $1 \le p < n$.
                \end{proposition}
                \begin{proof}
                  ($\impliedby$)
                  Let $r, s$ so that $n \mid (r - sk)$ for some integer $1 \le k < n$.
                  Then, $g^{r - sk} = e$ which implies that $g^{r} = g^{sk}$. Since $g^{r} \in \angles{g^{s}}$, the proposition follows.

                  ($\implies$)
                  Suppose we have such $r, s$ so that $\angles{g^{r}} \subset \angles{g^{s}}$.
                  Then, there is some $1 \le k < n$ so that $g^{r} = g^{sk}$.
                  Hence, it follows that $g^{r - sk} = e$ and thus $n \mid (r - sk)$.
                \end{proof}


          \item Let $H \le G$ and $\ord(g) = n$.
                \textbf{Prove}: If $g^{m} \in H$ and $m, n$ are coprime, then $g \in H$.

                \begin{proof}
                  Consider $\angles{g^{m}}$. Note that
                  \[ \abs{\angles{g^{m}}} = \frac{\ord(g)}{\gcd(m, n)}. \]
                  Since $\gcd(n, m) = 1$, we in fact have that $\abs{\angles{g^{m}}} = n$.
                  For this to be the case, it follows that $g \in H$, as $H$ must include all $n$ powers of $g$.
                \end{proof}

          \item Find all generators of $\ZZ$. Let $a$ be a group element that has infinite order.
                Find all generators of $\angles{a}$.

                All the generators of $\ZZ$ are $\set{1, -1}$.
                This is because $k\ZZ$ for $\abs{k} \ne 1$ we don't have 1 at all.

                Now let $a \in \ZZ$. The generators of $\angles{a}$ are precisely $\set{a, -a}$.
                This is because if $\abs{c} \ne \abs{a}$, we either have $c \mid a$ or $a \notin \angles{c}$.
                If $c \mid a$, then we in fact get a LARGER set than just $\angles{a}$, as $c \notin \angles{a}$.

        \end{enumerate}

  \item Choose \underline{two} to solve:

        \begin{enumerate}[start=1,label={(\alph*)}]
          \item Suppose that $\abs{G} = 24$ and that $G$ is cyclic.
                Let $a \in G$. If $a^{8} \ne e$ and $a^{12} \ne e$, show that $\angles{a} = G$.

                \begin{proof}
                  Suppose $\ord(a) < 24$ (this is equivalent).
                  This requires that $\ord(a) \mid 24$ and so $\ord(a) \in \set{1, 2, 3, 4, 6, 8, 12}$.
                  We know it is not one of $1, 8, 12$ by inspecting the given constraints.
                  The others are also impossible as $8, 12$ are multiples of them, and so also may not be identities.
                  Thus, $\ord(a) = 24$ and $a$ generates $G$.
                \end{proof}

          \item List all subgroups of $\ZZ_{30}$.

                \begin{itemize}
                  \item $\set{0}$.
                  \item $\set{0, 15}$.
                  \item $\set{0, 10, 20}$.
                  \item $\set{0, 5, 10, 15, 20, 25}$.
                  \item $\set{0, 2, 4, 6, 8, 10, 12, 14, 16, 18, 20, 22, 24, 26, 28}$.
                  \item $\set{0, 3, 6, 9, 12, 15, 18, 21, 24, 27}$.
                  \item $\set{0, 6, 12, 18, 24}$.
                  \item $\ZZ_{30}$.
                \end{itemize}

          \item List the cyclic subgroups of $U_{30} = \set{1, 7, 11, 13, 17, 19, 23, 29}$.

                \begin{itemize}
                  \item $\set{1}$.
                  \item $\set{1, 29}$.
                  \item $\set{1, 11}$.
                  \item $\set{1, 7, 13, 19}$.
                  \item $\set{1, 17, 19, 23}$.
                \end{itemize}

        \end{enumerate}

  \item Choose \underline{one} to solve:

        \begin{enumerate}[start=1,label={(\alph*)}]
          \item Suppose $\phi \from G \isoto H$.
                Prove that for $a, b \in G$, $a$ and $b$ commute if and only if $\phi(a)$ and $\phi(b)$ commute.

                \begin{proof}
                  ($\implies$)
                  Let $a, b$ commute. Then, $\phi(a)\phi(b) = \phi(ab) = \phi(ba) = \phi(b)\phi(a)$.

                  ($\impliedby$)
                  Now, the opposite. We see $\phi(ab) = \phi(a)\phi(b) = \phi(b)\phi(a) = \phi(ba)$.
                  This is possible since $\phi$ is necessarily bijective, and so we conclude $ab = ba$.
                \end{proof}

          \item Suppose $\phi \from G \isoto H$. Prove $G$ is cyclic if and only if $H$ is cyclic.

                \begin{proof}
                  ($G \implies H$)
                  Let $g \in G$ be a generator. Since $\ord(g) = \abs{G} = \abs{H}$, then $\phi$ preserves this order. Therefore, $H = \angles{\phi(g)}$ is cyclic.

                  ($H \implies G$)
                  There was nothing special about starting with $G$.
                  Since $\phi$ is an isomorphim, the same property holds in the opposite direction
                  and so for a generator $h \in H$ we have $\ord(\phi\inv(h)) = \abs{H} = \abs{G}$ and thus $G = \angles{\phi\inv(h)}$ is cyclic.
                \end{proof}

        \end{enumerate}


  \item Choose \underline{one} to solve:

        \begin{enumerate}[start=1,label={(\alph*)}]
          \item Show that $U_{8} \iso U_{12}$.

                The isomorphism $\phi \from U_{8} \to U_{12}$ that sends $1 \to 1$, $3 \to 5$, $5 \to 7$ and $7 \to 11$ works perfectly fine.
                This is because every element is still self-inverse, and $\phi(3 \cdot 5) = \phi(7) = 11$ and $35 \equiv 11 \pmod{12}$.
                By inspection, the rest of the pairwise operations also works.

          \item Show that $U_{8} \not\iso U_{10}$.

                There is no element of order 4 in $U_{8}$.
                Since isomorphisms preserve order, there is no valid $v \in U_{8}$ so that $i(v) = 3$.
        \end{enumerate}

  \item Textbook 17.9

        Consider the homomorphism $\gamma \from \ZZ_{12} \to \ZZ_{18}$ given by $a \mapsto 6a$.

        \begin{enumerate}[start=1,label={(\alph*)}]
          \item How are 2 and 10 related in the domain? They are inverses.

          \item How are they related in the image? They are still inverses. That is, $6(2) + 6(10) = 72 = 18 \cdot 4 = 0$.

          \item Repeat for 4, 8. They are \emph{\underline{\textbf{still}}} inverses in both the domain and codomain.

          \item Is this surprising?

                No! It is a very trivial theorem that inverses are sent to inverses under homomorphisms.
        \end{enumerate}


  \item Textbook 18.13

        Complete the proof of 18.7 by proving the first implication.

        \begin{proposition}{}{}
          For $\lambda \from G \to H$, if $\ker \lambda$ is trivial then $\lambda$ is injective.
        \end{proposition}

        Luckily I literally proved this in my notes like a month ago:

        \begin{proof}
          Suppose $\lambda$ has trivial kernel.
          Let $a, b$ so that $\lambda(a) = \lambda(b)$.
          Then, we have that $e = \lambda(b)\lambda(a)\inv$ and so $\lambda(b a\inv) = e$.
          Therefore, $a\inv = b\inv$ and so $a = b$.
        \end{proof}


  \item Textbook 18.18

        \begin{itemize}
          \item Find all endomorphisms of $\ZZ_{18}$.

                The easy ones are $a \mapsto ak$ for all integers $k$.

                Nonlinear transformations cannot be homomorphisms since $T(a + b) = T(a) + T(b)$ is \emph{the definition}
                of a linear transformation. It also cannot be affine since then the identity may not be preserved.

          \item Find the automorphisms now.

                For above, an endomorphism $f_{k}$ is an isomorphism if it is injective (surjectivity follows)
                and so for $\gcd(k, 18) = 1$ we have $f_{k}(a) = f_{k}(b) \implies ka = kb$.
                Through modular arithmetic we find then $18 \mid (ka - kb)$.
                Since 18 is coprime to $k$, it follows (by Euclid or something) that $18 \mid (a - b)$ and so $a = b$ in $\ZZ_{18}$.

                If the $k$ is not coprime to 18, then it is easy to find elements that are sent to the same spot in the image.
        \end{itemize}


  \item Textbook 19.22

        \begin{proposition}{}{}
          Let $H, K \le G$. If $aH \subset bK$ for some $a, b \in G$, then $H \subset K$.
        \end{proposition}

        \begin{proof}
          Suppose $aH \subset bK$ for some $a, b \in G$.
          Then, this means that $a \in bK$ and so $aH \subset aK$.
          By left-multiplying by $a\inv$ we see that $H \subset K$.
        \end{proof}

\end{enumerate}


\end{document}
