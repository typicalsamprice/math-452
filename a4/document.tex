% -*- compile-command: "latexmk -pdf document.tex" -*-
\documentclass{article}

\author{Sam Price}
\title{Math 452: Assignment 4}

\newif\ifprinted%
\printedtrue%
% Most taken from: https://github.com/SeniorMars/dotfiles/blob/main/latex_template/preamble.tex
\usepackage{geometry}

\usepackage[english]{babel}
\usepackage[T1]{fontenc}
\usepackage[utf8]{inputenc}
\usepackage{lmodern}

% Use the command \doublespacing if needed
\usepackage{setspace}

\geometry{a4paper, margin=1in}

\usepackage{mathtools}
\usepackage{amssymb,amsfonts,amsthm}
\usepackage{bm}

\usepackage{xfrac}
\usepackage[makeroom]{cancel}

% Use \begin{enumerate}[start=x,label={Q\arabic*)}] for example
\usepackage{enumitem}

\usepackage{xcolor}

\usepackage{nameref}

% Important options
% in envs, use options like [baseline=x] to center on row x (or special opts t/[c]/b without 'baseline')
\usepackage{nicematrix}
\NiceMatrixOptions{cell-space-limits = 1pt}

\usepackage{booktabs}

\usepackage{tikz}
\usepackage{tikz-cd}
\usepackage{tikzsymbols}

\usepackage{pdfpages}

\setlength{\parindent}{1cm}

\DeclarePairedDelimiter{\pars}{\lparen}{\rparen}
\DeclarePairedDelimiter{\abs}{\lvert}{\rvert}
\DeclarePairedDelimiter{\bracks}{\lbrack}{\rbrack}
\DeclarePairedDelimiter{\angles}{\langle}{\rangle}
\DeclarePairedDelimiter{\set}{\lbrace}{\rbrace}
\DeclarePairedDelimiter{\floor}{\lfloor}{\rfloor}
\DeclarePairedDelimiter{\ceil}{\lceil}{\rceil}

\let\class\bracks% % Equivalence class redef


% Special Groups
\DeclareMathOperator{\Fr}{Fr} % Frame, Math 464
\DeclareMathOperator{\Aut}{Aut}
\DeclareMathOperator{\Mor}{Mor}
\DeclareMathOperator{\Hom}{Hom}
\DeclareMathOperator{\Orth}{O}
\DeclareMathOperator{\SO}{SO}
\DeclareMathOperator{\GL}{GL}
\DeclareMathOperator{\Mat}{Mat}
\DeclareMathOperator{\PGL}{PGL}
\DeclareMathOperator{\SL}{SL}
\DeclareMathOperator{\Sym}{Sym}
\DeclareMathOperator{\Alt}{Alt}
\DeclareMathOperator{\Dih}{Dih}

\DeclareMathOperator{\Fix}{Fix}
\DeclareMathOperator{\Support}{Support}

\DeclareMathOperator{\tr}{tr}

\DeclareMathOperator{\im}{im}
\DeclareMathOperator{\rank}{rank}
\DeclareMathOperator{\Span}{span}

\DeclareMathOperator{\ord}{ord}
\DeclareMathOperator{\lcm}{lcm}

\DeclareMathOperator{\cl}{cl}
\DeclareMathOperator{\Int}{Int}
\DeclareMathOperator{\Ext}{Ext}
\DeclareMathOperator{\Bd}{Bd}

\DeclareMathOperator{\D}{D\!}% Multidim derivative

\DeclareMathOperator{\cis}{cis}

\newcommand{\from}{\colon}
\newcommand{\injto}{\hookrightarrow}
\newcommand{\surjto}{\twoheadrightarrow}
\newcommand{\normalto}{\triangleleft}
\newcommand{\iso}{\simeq}
\newcommand{\isoto}{\xrightarrow{\sim}}

\renewcommand{\vec}[1]{\bm{#1}}

\newcommand{\RR}[1][]{\mathbb{R}^{#1}}
\newcommand{\ZZ}[1][]{\mathbb{Z}^{#1}}
\newcommand{\CC}[1][]{\mathbb{C}^{#1}}
\newcommand{\QQ}[1][]{\mathbb{Q}^{#1}}
\newcommand{\PP}[1][]{\mathbb{P}^{#1}}
\newcommand{\NN}[1][]{\mathbb{N}^{#1}}

\renewcommand{\qedsymbol}{$\blacksquare$}

\newcommand{\ol}{\overline}

\newcommand{\id}[1][]{\mathrm{id}_{#1}}

\newcommand{\inv}{^{-1}}

\let\epsilon\varepsilon%

%%%%%%%%%%%%%%%%%%%%%
%%% THEOREM BOXES %%%
%%%%%%%%%%%%%%%%%%%%%
\usepackage[most,many,breakable]{tcolorbox}
\tcbuselibrary{theorems,skins,hooks}

% Stolen from: https://tex.stackexchange.com/a/330460
\makeatletter
\renewenvironment{proof}[1][\proofname]{\par
  \pushQED{\qed}%
  \normalfont \topsep6\p@\@plus6\p@\relax
  \trivlist
  \item[%
    \hskip\labelsep
    \normalfont\bfseries % was \itshape
    #1%
    \@addpunct{.}% remove this if you don't want punctuation
  ]\ignorespaces
}{%
  \popQED\endtrivlist\@endpefalse
}
\let\qed\relax % avoid a warning
\DeclareRobustCommand{\qed}{%
  \ifmmode \mathqed
  \else
    \leavevmode\unskip\penalty\@M\hbox{}\nobreak\hspace{.5em minus .1em}% was \hfill
    \hbox{\qedsymbol}%
  \fi
}
\makeatother

\ifprinted%
  \colorlet{thmbgcol}{white}
  \colorlet{lembgcol}{white}
  \colorlet{corbgcol}{white}
  \colorlet{propbgcol}{white}
  \colorlet{exbgcol}{white}
  \colorlet{defbgcol}{white}

  \colorlet{qheadcol}{black!20!white}
  \colorlet{qheadtxtcol}{black!90}

  \colorlet{exhlcol}{black}
  \colorlet{prophlcol}{black}
  \colorlet{corhlcol}{black}
  \colorlet{thmhlcol}{black}
  \colorlet{lemhlcol}{black}
  \colorlet{defhlcol}{black}
\else
  \definecolor{thmbgcol}{HTML}{aec1f9}
  \definecolor{corbgcol}{HTML}{b599f7}
  \definecolor{propbgcol}{HTML}{c9f7aa}
  \definecolor{exbgcol}{HTML}{f7c479}
  \colorlet{defbgcol}{red!7}

  \definecolor{qheadcol}{HTML}{182959}
  \colorlet{qheadtxtcol}{white}

  \definecolor{exhlcol}{HTML}{604419}
  \definecolor{prophlcol}{HTML}{1e631a}
  \definecolor{corhlcol}{HTML}{2d1760}
  \definecolor{thmhlcol}{HTML}{142c72}
  \colorlet{defhlcol}{red!50!black}

  \definecolor{lembgcol}{HTML}{e998f2}
  \definecolor{lemhlcol}{HTML}{791684}
\fi

\newtcbtheorem[number within = section]{theorem}{Theorem}{
  enhanced, breakable, colback = thmbgcol!20,
  frame hidden, boxrule = 0sp, borderline west = {2pt}{0pt}{thmhlcol},
  sharp corners, detach title, before upper = \tcbtitle\par\smallskip,
  coltitle = thmhlcol, fonttitle = \bfseries,
  description font = \mdseries, separator sign none, segmentation style = {solid, thmhlcol}
}{th}

\newtcbtheorem[number within = section]{lemma}{Lemma}{
  enhanced, breakable, colback = lembgcol!20,
  frame hidden, boxrule = 0sp, borderline west = {2pt}{0pt}{lemhlcol},
  sharp corners, detach title, before upper = \tcbtitle\par\smallskip,
  coltitle = lemhlcol, fonttitle = \bfseries,
  description font = \mdseries, separator sign none, segmentation style = {solid, lemhlcol}
}{lem}

\newtcbtheorem[number within = section]{corollary}{Corollary}{
  enhanced, breakable, colback = corbgcol!20,
  frame hidden, boxrule = 0sp, borderline west = {2pt}{0pt}{corhlcol},
  sharp corners, detach title, before upper = \tcbtitle\par\smallskip,
  coltitle = corhlcol, fonttitle = \bfseries,
  description font = \mdseries, separator sign none, segmentation style = {solid, corhlcol}
}{cor}

\newtcbtheorem[number within = section]{proposition}{Proposition}{
  enhanced, breakable, colback = propbgcol!25,
  frame hidden, boxrule = 0sp, borderline west = {2pt}{0pt}{prophlcol},
  sharp corners, detach title, before upper = \tcbtitle\par\smallskip,
  coltitle = prophlcol, fonttitle = \bfseries,
  description font = \mdseries, separator sign none, segmentation style = {solid, prophlcol}
}{prop}

\newtcbtheorem[number within = section]{example}{Example}{
  enhanced, breakable, colback = exbgcol!20,
  frame hidden, boxrule = 0sp, borderline west = {2pt}{0pt}{exhlcol},
  sharp corners, detach title, before upper = \tcbtitle\par\smallskip,
  coltitle = exhlcol, fonttitle = \bfseries,
  description font = \mdseries, separator sign none, segmentation style = {solid, exhlcol}
}{prop}

\newtcbtheorem[number within = section]{definition}{Definition}{
  enhanced, breakable, colback = defbgcol,
  frame hidden, boxrule = 0sp, borderline west = {2pt}{0pt}{defhlcol},
  sharp corners, detach title, before upper = \tcbtitle\par\smallskip,
  coltitle = defhlcol, fonttitle = \bfseries,
  description font = \mdseries, separator sign none, segmentation style = {solid, exhlcol}
}{def}

\makeatletter
\newtcbtheorem{question}{Question}{enhanced,
    breakable,
    colback=white,
    colframe=qheadcol,
    attach boxed title to top left={yshift*=-\tcboxedtitleheight},
    fonttitle=\bfseries,
    coltitle=qheadtxtcol,
    title={#2},
    boxed title size=title,
    boxed title style={%
            sharp corners,
            rounded corners=northwest,
            colback=qheadcol,
            boxrule=0pt,
        },
    underlay boxed title={%
            \path[fill=tcbcolframe] (title.south west)--(title.south east)
            to[out=0, in=180] ([xshift=5mm]title.east)--
            (title.center-|frame.east)
            [rounded corners=\kvtcb@arc] |-
            (frame.north) -| cycle;
        },
    #1
}{}
\makeatother


\usepackage{changepage}
\newenvironment{remark}{\begin{adjustwidth}{1cm}{}\textbf{\underline{Remark}}.}{\end{adjustwidth}}


\begin{document}

\maketitle

\begin{enumerate}[start=0,label={RELABEL}]
  \item[14.8\rparen] Explain why $\QQ$ is not cyclic.

        The rationals are not cyclic because there is no $p/q$ such that for \emph{every} $s/t$ there is an integer $k$ so that $kp/q = s/t$.
        For instance, since we take $(p, q)$ to be coprime, simply take $(p, q^{2})$. If $q = 1$, then of course an integer may not generate all rationals.

  \item[14.15\rparen] Find all generators of $\ZZ_{3} \times \ZZ_{4}$.

        The generators are
        \[ (1, 2), (2, 1), (1, 3), (2, 3). \]

  \item[\textbf{14.21}\rparen] Let $H \le G$ and $g \in G$ fixed.
        Prove that if $G$ is cyclic, then the conjugate of $H$ w.r.t. $g$ is cyclic.

        \begin{proof}
          Since $gHg\inv$ is a subgroup of $G$ and $G$ itself is cyclic,
          we know $gHg\inv$ must be cyclic as well.
        \end{proof}

  \item[16.12\rparen] Let $G$ be a group and fix $g \in G$.
        Define $\lambda \from G \to G$ so that $\lambda(a) = gag\inv$. Prove $\lambda$ is an isomorphism.

        \begin{proof}
          Note that $\lambda(ab) = gabg\inv = ga (gg\inv) bg = \lambda(a)\lambda(b)$.
          Since $\lambda$ is a homomorphism, we need only show that $\ker \lambda$ is trivial now,
          as then $\lambda$ would be injective and so surjective.
          Suppose there is some $a \in G$ so that $\lambda(a) = e$.
          Then, $gag\inv = e$ which implies that $ga = g$ and so $a = e$.
        \end{proof}

  \item[\textbf{16.16}\rparen] Determine if each pair of groups is isomorphic.

        \begin{lemma}{}{zn-zm-iso-cyc}
          Let $A, B$ be cyclic groups of order $m, n$ respectively.
          Then, $A \times B$ is cyclic if and only if $\gcd(m, n) = 1$.
        \end{lemma}
        \begin{proof}
          ($\impliedby$)
          Suppose $\gcd(m, n) = 1$ and $a, b$ be generators of $A$ and $B$ respectively.
          $A \times B$ is cyclic only if there is an element of order $mn$ within it.
          We claim that $(a, b)$ is such an element.
          Clearly, its order is some divisor of $mn$, that much is clear by simple inspection.
          Suppose its order is $k < mn$. This means that $a^{k} = b^{k} = e$,
          but then $m \mid k$ and $n \mid k$ which is a contradiction as $\lcm(m, n) = mn$.% Comment for footnote spacing
          \footnote{Yes, I get the abuse of notation here by not distinguishing between $e_{A}, e_{B}$ and $e_{A \times B}$.}
          Hence, $(a, b)$ has order $mn$ and so $A \times B$ is cyclic.

          ($\implies$)
          Suppose now $\gcd(m, n) > 1$. Let $a^{k}, b^{j}$ be any elements of $A, B$ for generators $a, b$.
          Thus $\lcm(m, n) < mn$ and so
          \[ \pars{a^{k}, b^{j}}^{\lcm(m, n)} = e. \]
          Since every element has order less than $mn$, $A \times B$ is not cyclic.
        \end{proof}


        \begin{enumerate}[start=1,label={(\alph*)}]
          \item Yes.

          \item No.

          \item No. First is not cyclic, second one is.
        \end{enumerate}

        Addendum: The reason behind the one-word answers is that if $A$ and $B$ are both cyclic groups, and $\abs{A} = \abs{B}$,
        then it is trivial to show that $A \iso B$.
        Sketch: Send generator powers to each other.

\newpage% Formatting

  \item[16.19\rparen] Let $\theta \from \ZZ_{12} \to \ZZ_{12}$ defined by $n \mapsto 4n$.

        \begin{itemize}
          \item Verify $\theta$ is operation-preserving.

                Let $a, b \in \ZZ_{12}$. Then,
                \[ \theta(a + b) = 4(a + b) = 4a + 4b = \theta(a) + \theta(b). \]

          \item Is $\theta$ an isomorphism?

                No, as $\ker \theta$ is nontrivial. Note $3 \in \ker \theta$, which is enough.
        \end{itemize}


  \item[17.6\rparen] Let $f \from \ZZ_{15} \to \ZZ_{20}$ be defined by $n \mapsto 4n$. Show $f$ is a homomorphism.

        Firstly, $f(0) = 0$. Let $a, b \in \ZZ_{15}$.
        We find that $f(a + b) = 4(a + b) = 4a + 4b = f(a) + f(b)$.
        We don't bother checking if they are equal when crossing modulo-indices since we can take equivalence classes and addition is fine across those.
        That is, $[a + b] = [a] + [b]$.

  \item[17.7\rparen] For the same $f$ as in the previous question:

        \begin{enumerate}[start=1,label={(\alph*)}]
          \item Find $\ord(7)$ and $\ord(f(7))$.

                Easily $\ord(7) = 15$ since they are coprime.
                Then, we know $\ord(f(7)) \mid 15$.
                In fact, we know with this specific $f$ every order of $f(n)$ must divide 5. So, since it is not in $\ker f$, it is 5 exactly.

          \item Same for 6.

                Order 5 for both. See reasoning for the latter above.

          \item Run it back for 10.

                $\ord(10) = 3$ and has order 0 under $f$ since it is in the kernel of $f$.

          \item Is this surprising? Explain.

                Not really. The $n \mapsto 4n$ map really trivializes it as either the order of something under $f$ is 0 (when $a \in \ker f$)
                or must be 5 itself since $20/4 = 5$ is prime.
        \end{enumerate}


  \item[17.10\rparen] Consider $\lambda \from U_{13} \to U_{13}$ given by $a \mapsto a^{3}$.

        \begin{enumerate}[start=1,label={(\alph*)}]
          \item How are 2 and 7 related in the domain of $U_{13}$?

                They are related by the fact $7 \cdot 2 = 1$.

          \item Their images?

                In this case $\lambda(7) \lambda(2) = 7^{3}2^{3} = 1 = \lambda(2 \cdot 7)$.

          \item Repeat for elements $4, 10 \in U_{13}$.

                We see $4 \cdot 10 = 1$. Their images hold that $\lambda(4)\lambda(10) = 4^{3}10^{3} = 1 = \lambda(4 \cdot 10)$.

          \item Is this surprising? Explain.

                No, since homomorphisms must preserve inverses and group operations.
        \end{enumerate}

\end{enumerate}


\end{document}
