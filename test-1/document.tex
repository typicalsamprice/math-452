% -*- compile-command: "latexmk -pdf document.tex" -*-
\documentclass{article}

\input{template}
\togglefalse{paper}

\input{theoremboxes}

\author{Sam Price}
\date{October 2, 2024}
\title{Test 1}

\begin{document}

\maketitle

\begin{enumerate}
  \item \begin{enumerate}
          \item Let $a, n$ be positive integers with greatest common divisor $d$. Show that $ax = 1$ has a solution in $\bbZ_{n}$ if and only if $d = 1$.
                \begin{proof}
                  Let $a, n$ be positive integers.
                  Let us first suppose that they are coprime, and note that then there is a solution $ax + bn = 1$
                  in integers. In $\bbZ_{n}$, this is \emph{identically} our solution for $ax = 1$ since $bn \equiv 0 \pmod{n}$.

                  Now suppose that $\gcd(a, n) = d > 1$. Then by B\'ezout we have that for
                  \emph{every pair of integers} $x, y$
                  \[ d \mid (ax + ny) \]
                  which in $\bbZ_{n}$ looks like $d \mid ax$. However, since $d > 1$, we know that $ax \ne 1$ for all $x$
                  because $d \nmid 1$.
                \end{proof}

          \item Show $\gcd(a, bc) = 1$ if and only if $\gcd(a, b) = \gcd(a, c) = 1$.
                \begin{proof}
                  Suppose first that $\gcd(a, bc) = 1$. Then we note that there are integers $x, y$ so that
                  \[ ax + bcy = 1. \]
                  Furthermore, if we take $by$ and $cy$ as their own integers, it is clear that we have
                  a solution to $ax + bz = 1$ and $ax + cz = 1$ which may only be possible
                  if $\gcd(a, b) = \gcd(a, c) = 1$ again by B\'ezout.

                  Now suppose that only $\gcd(a, b) = \gcd(a, c) = 1$.
                  We see then that there are integers, $w, x, y, z$ so that
                  \[ aw + by = ax + cz = 1. \]
                  Multiplying the leftmost side of this equation by the center (which is 1), we find that
                  \[ (aw + by)(ax + cz) = (a)(awx + byx + wcz) + (bc)(yz) = 1. \]
                  Thus we have integers so that $ax + bcy = 1$
                  for the substitutions $x = awx + byx + wcz$ and $y = yz$.
                  It follows that $\gcd(a, bc) = 1$.
                \end{proof}
        \end{enumerate}

  \item \begin{enumerate}
          \item Verify $4 \in \bbZ_{5}$ is self-inverse. Well, $4^{2} = 16 = 3 \cdot 5 + 1 = 1$.
          \item $6 \in \bbZ_{7}$ has $36 = 35 + 1 = 1$.
          \item $9 \in \bbZ_{10}$ has $81 = 80 + 1 = 1$.
          \item $14 \in \bbZ_{15}$ has $196 = 195 + 1 = 1$.
          \item Let $m - 1 \in \bbZ_{m}$. Then, $(m - 1)^{2} = m^{2} - 2m + 1 \equiv 1 \pmod{m}$.
        \end{enumerate}


  \item Let $a, b \in G$. Show that $ab = ba$ if and only if $a\inv b\inv = b \inv a \inv$.
        \begin{proof}
          Let $a, b \in G$. Assume first that $ab = ba$.
          Because $a, b$ are generic this requires $G$ be abelian,
          the conclusion follows easily since $a\inv, b\inv \in G$ by definition.

          Now, suppose $a\inv b \inv = b \inv a \inv$.
          Right-multiply both sides by $ab$ (as $ab = (b \inv a \inv)\inv$), and we find that
          $a\inv b\inv ab = 1$. If we left-multiply both sides by $ba = (a\inv b\inv)\inv$ we find that
          $ab = ba$, which is what we wanted to show.
        \end{proof}

        This feels like it could be said that commutativity is \emph{necessary} for either to be true, and thus
        assuming one implies the other.

  \item \begin{enumerate}
          \item The set $\set{1, 5, 7, 11} \subset \bbZ_{12}$ under multiplication.

                This set \emph{is} a group.
                Each element is self-inverse and it is closed.
                Further, since it is solely composed of self-inverse elements, it is abelian.

          \item The set $\set{0, 5, 10, 15} \subset \bbZ_{20}$ under addition.

                We have identity and associativity by (nearly) definition.
                The closure is merely by inspection, and inverses are clear to see as well.
                Thus, this is a group as well.
        \end{enumerate}

  \item List all elements of $U_{10}$. Rephrased, is $\generator{3} = U_{10}$ and/or $\generator{9} = U_{10}$?

        Firstly, $U_{10} = \set{1, 3, 7, 9}$.
        This is clearly all since $\phi(10) = (2 - 1)(5 - 1) = 4$, and so $\abs{U_{10}} = 4$.
        Less flashy, but it's easy enough by inspection to see these are the only positive integers
        under ten coprime to it.

        The group cannot be generated by 9.
        This is easily seen since 9 is self-inverse: any power of itself is either 1 or 9.

        For 3, we may explicitly see that $3^{2} = 9$ and $3^{3} = 27 = 7$. Then, $\ord(3) = 4$ and
        so we have the identity generated. So, $\generator{3} = U_{10}$.

  \item Consider $U_{7}, U_{9}, U_{14}, U_{18}$ and $\bbZ_{2} \times \bbZ_{3}$.
        \begin{enumerate}
          \item Verify each has 6 elements.

                For the unit groups (is there some more established name?),
                we may use Euler's totient function to count their elements:
                \begin{itemize}
                  \item $\phi(7) = 7 - 1 = 6$.
                  \item $\phi(9) = \phi(3^{2}) = 3^{2 - 1}(3 - 1) = 3 \cdot 2 = 6$.
                  \item $\phi(14) = \phi(2 \cdot 7) = (2 - 1)(7 - 1) = 6$.
                  \item $\phi(18) = \phi(2 \cdot 3^{2}) = 3^{2 - 1}(3 - 1)(2 - 1) = \phi(9) = 6$.
                \end{itemize}
                For $\bbZ_{2} \times \bbZ_{3}$, we may note that $\abs{G \times H} = \abs{G} \cdot \abs{H}$, and so by
                simply counting we find that $\abs{\bbZ_{2} \times \bbZ_{3}} = \abs{\bbZ_{2}} \cdot \abs{\bbZ_{3}}$, which
                is clearly $2 \cdot 3 = 6$.

          \item Determine if each group is cyclic, and if so give a generator.

                \begin{itemize}
                  \item $U_{7} = \generator{3} = \set{3, 2, 6, 4, 5, 1}$
                  \item $U_{9} = \generator{2} = \set{2, 4, 8, 7, 5, 1}$
                  \item $U_{14} = \generator{3} = \set{3, 9, 13, 11, 5, 1}$
                  \item $U_{18} = \generator{5} = \set{5, 7, 17, 13, 11, 1}$
                  \item The product is generated by
                        $\generator{(1, 2)} = \set{(1, 2), (0, 1), (1, 0), (0, 2), (1, 1), (0, 0)}$.
                \end{itemize}

          \item Is every six-element group cyclic? No, consider $S_{3}$ as an acyclic group of order 6.
        \end{enumerate}


  \item Let $G$ be a finite group of order $n$.
        Let $G_{k} = \set*{ x \in G : x^{k} = e }$.
        Show that $\abs{\comp{G_{2}}} \equiv 0 \pmod{2}$ and $\abs{G_{3}} \equiv 1 \pmod{2}$.


        \begin{proof}
            First, let $\varphi \from G \to G$ be the involution $g \mapsto g\inv$.
            The fixed points of $\varphi$ is in fact our set $G_{2}$.
            Since $\abs{\Fix \varphi} \equiv \abs{G} \pmod{2}$ by virtue of $\varphi$ being involutory,
            we know that our desired integer
            \[
              \abs{\comp{G_{2}}} = \abs{G \setminus \Fix \varphi} = \abs{G} - \abs{\Fix \varphi} \equiv 0 \pmod{2}.
            \]

            Now, consider $g \ne e$ for $g^{3} = e$. Then we also note that $\pars{g\inv}^{3} = e\inv = e$.
            Since $g^{2} = g\inv$ we know that $g \ne g\inv$, and so such elements of $G_{3}$ come in pairs.
            Then, we add in the identity itself to get an odd number of elements in $G_{3}$.
        \end{proof}

\end{enumerate}


\end{document}
